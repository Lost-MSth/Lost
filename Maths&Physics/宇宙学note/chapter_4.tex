\subsection{简单宇宙模型}
\begin{equation}
	H^2 = \qty(\frac{\dot{a}}{a})^2 = \frac{8 \pi G}{3} \rho_{\text{tot}} - \frac{k c^2}{a^2}
\end{equation}
\begin{cnote}
	流体和物态方程对不同介质是分别成立的,但Friedmann方程的$\rho_{\text{tot}}$是所有介质的质量密度之和,一般分别讨论只含某种介质的情况
\end{cnote}

\subsubsection{空的宇宙$\rho_{\text{tot}}=0$}
\par 
此时Friedmann方程变为$H^2 = \qty(\frac{\dot{a}}{a})^2 = - \frac{k c^2}{a^2}$,则$\dot{a} = \sqrt{-k} c$,现在方程只有曲率了
\begin{itemize}
	\item[1)] $k > 0$ 无意义
	\item[2)] $k = 0, \dot{a} = 0, a = \text{const}$ 平直宇宙不膨胀,无物质,是平庸解
	\item[3)] $k < 0, a \propto t \stackrel{a(t_0)=1}{\Longrightarrow} a(t) = \frac{t}{t_0}, t_0 = \qty(\sqrt{-k} c)^{-1}$ 描述了一个虚无(empty)、负曲率、膨胀的宇宙,Milne宇宙(1930年E.A.Milne)
\end{itemize}

\subsubsection{$k=0$仅有一类宇宙介质}
\par 
此时Friedmann方程变为$\qty(\frac{\dot{a}}{a})^2 = \frac{8 \pi G}{3} \rho_{\text{tot}}$

\begin{itemize}
	\item[1)] 仅含有宇宙学常数$\rho_{\Lambda} = \frac{\Lambda}{8 \pi G} \Rightarrow H^2(t) = \qty(\frac{\dot{a}}{a})^2 = \frac{\Lambda}{3}, \Lambda > 0$,这说明尺度因子按指数变化$a(t) = \ee^{H_0 (t - t_0)}$ \\
	de Sitter宇宙(1917年W.de Sitter),没有物质,不能描述当前宇宙,但可以描述宇宙早期的暴涨阶段
	
	\item[2)] 对尘埃和辐射 \\
	有$\rho \propto a^{-3(1+\omega)}$,代入Friedmann方程得到$\qty(\frac{\dot{a}}{a})^2 \propto a^{-3(1+\omega)}$ $\Rightarrow$ $a(t) = \qty(\frac{t}{t_0})^{\frac{2}{3(1+\omega)}}, \rho(a(t)) = \rho(t) \propto t^{-2}$ \\
	具体地:(Einstein-de Sitter模型,1932年)
	\begin{itemize}
		\item[a.] 对尘埃,$\omega = 0$,$a(t) = \qty(\frac{t}{t_0})^{\frac{2}{3}}$,$\rho(t) = \rho_0 \qty(\frac{t}{t_0})^{-2}$
		\item[b.] 对辐射,$\omega = \frac{1}{3}$,$a(t) = \qty(\frac{t}{t_0})^{\frac{1}{2}}$,$\rho(t) = \rho_0 \qty(\frac{t}{t_0})^{-2}$
	\end{itemize}
\end{itemize}

\begin{table}[!h]
	\centering
	\scriptsize
	\resizebox{\linewidth}{!}{
	\begin{tabular}{|c|ccc|c|}
		\hline
		& \multicolumn{3}{c|}{$k=0$}                                                                                                                                                                                                 & $k<0$ \\ \hline
		& \multicolumn{1}{c|}{de Sitter}                                                   & \multicolumn{1}{c|}{Einstein-de Sitter}                                      & 纯辐射                                                      & Milne         \\ \hline
		宇宙组成   & \multicolumn{1}{c|}{\begin{tabular}[c]{@{}c@{}}只含$\Lambda$\\ $\omega=-1$\end{tabular}} & \multicolumn{1}{c|}{\begin{tabular}[c]{@{}c@{}}只含尘埃\\ $\omega=0$\end{tabular}} & \begin{tabular}[c]{@{}c@{}}只含辐射\\ $\omega=\frac{1}{3}$\end{tabular} & 只有负曲率         \\ \hline
		尺度因子 $a(t)$  & \multicolumn{1}{c|}{$a=\ee^{H_0(t - t_0)}$}                                                         & \multicolumn{1}{c|}{$a = \qty(\frac{t}{t_0})^{\frac{2}{3}}$}                                                       &      $a = \qty(\frac{t}{t_0})^{\frac{1}{2}}$       &    $a = \frac{t}{t_0}$   \\ \hline
		哈勃参数   & \multicolumn{1}{c|}{$H(t)=H(t_0)=\sqrt{\frac{\Lambda}{3}}$}                                                           & \multicolumn{1}{c|}{$H(t) = \frac{2}{3t}$}          &        $H(t) = \frac{1}{2t}$               &   $H(t) = \frac{1}{t}$   \\ \hline
		膨胀速度$v \propto \dot{a}$   & \multicolumn{1}{c|}{$v \propto \ee^{H_0 t}$}                                                           & \multicolumn{1}{c|}{$v \propto t^{-\frac{1}{3}}$}                                                       &               $v \propto t^{-\frac{1}{2}}$     &   $v = \text{const}$     \\ \hline
		减速参数$q = \frac{-\ddot{a}}{a H^2}$ & \multicolumn{1}{c|}{-1(加速)}                                                           & \multicolumn{1}{c|}{$\frac{1}{2}$(减速)}                                                       &    1(减速)            &    0(匀速自由膨胀)   \\ \hline
		$\rho(a)$ & \multicolumn{1}{c|}{常数$\frac{\Lambda}{8 \pi G}$}                                                           & \multicolumn{1}{c|}{$\propto a^{-3} \propto t^{-2}$}                                                       &                      $\propto a^{-4} \propto t^{-2}$    &   $a^{-2}$     \\ \hline
	\end{tabular}}
	\caption{简单宇宙模型表}
\end{table}

\subsection{多分量共存的宇宙学模型}
\subsubsection{平直时空,只含物质和辐射}
\par 
流体方程对不同分量分别成立,$\rho_{\text{mat}} \propto a^{-3}, \rho_{\text{rad}} \propto a^{-4}$,Friedmann方程中$\rho_{\text{tot}} = \rho_{\text{mat}} + \rho_{\text{rad}} \quad (\Lambda = 0)$难以解析求解
\par 
策略:考虑某一主导分量
\begin{itemize}
	\item[1)] 辐射主导$\rho_{\text{tot}} \simeq \rho_{\text{rad}}$,此时尺度因子行为$a(t) \propto t^{\frac{1}{2}}, \rho_{\text{rad}} \propto t^{-2} \Rightarrow \rho_{\text{mat}} \propto a^{-3} \propto t^{-\frac{3}{2}}$ \\
	随$t$增加,物质的质量密度衰减比辐射慢
	
	\item[2)] 物质主导$\rho_{\text{tot}} \simeq \rho_{\text{mat}}$,此时尺度因子行为$a(t) \propto t^{\frac{3}{2}}, \rho_{\text{mat}} \propto t^{-2} \Rightarrow \rho_{\text{rad}} \propto a^{-4} \propto t^{-\frac{8}{3}}$ \\
	仍然是辐射衰减快
\end{itemize}
\par 
故辐射主导不稳定而物质主导稳定,最终物质占主导地位

\subsubsection{更一般的情况}
\par
对于更一般的多分量共存宇宙,根据密度参数之间的关系(主要是$\Omega_0$和$\Omega_{\Lambda, 0}$的关系)
\par 
Friedmann方程$H^2 = \frac{8 \pi G}{3} (\rho + \rho_{\Lambda}) - \frac{k c^2}{a^2}, 1 = \frac{8 \pi G}{3 H^2} (\rho + \rho_{\Lambda}) - \frac{k c^2}{H^2 a^2} \Rightarrow 1 = \frac{\rho + \rho_{\Lambda}}{\rho_c} - \frac{k c^2}{a^2 H^2} \equiv \Omega + \Omega_{\Lambda} + \Omega_{k}$

\begin{itemize}
	\item 分类一:开放、封闭、平直 \\ 
	\cword[blue]{$\Omega_{\Lambda, 0} = 1 - \Omega_{0} + \frac{k c^2}{H_0^2}$},当$\Omega_{\Lambda, 0} = 1 - \Omega_{0}$时$k = 0$
	
	\item 分类二:由加速度方程$\frac{\ddot{a}}{a} = - \frac{4 \pi G}{3} \qty(\rho + \frac{3p}{c^2}) + \frac{\Lambda}{3}$以及减速参数定义$q = - \frac{\ddot{a}}{a H^2}$,可以得到$q H^2 = \frac{4 \pi G}{3} \qty(\rho + \frac{3p}{c^2}) - \frac{\Lambda}{3}$ \\
	考虑当前时刻(物质主导)$q_0^2 H_0^2 \simeq \frac{4 \pi G}{3} \rho_{\text{mat}, 0} - \frac{\Lambda}{3}$ \\ 故\cword[red]{$q_0 = \frac{4 \pi G}{3 H_0^2} \rho_0 - \frac{\Lambda}{3 H_0^2} = \frac{1}{2} \Omega_{0} - \Omega_{\Lambda, 0}$} \\
	若进一步考虑$k=0$情况,Friedmann方程退化为$\Omega + \Omega_{\Lambda} = 1, q_0 = \frac{3}{2} \Omega_0 - 1$,当$\Omega_{0} = \frac{2}{3}$时$q_0 = 0 \Rightarrow \Omega_{\Lambda, 0} > \frac{1}{3}$则加速膨胀
	
	\item 其它分类:1. 宇宙初始时刻\cword[magenta]{有无大爆炸};2. 宇宙\cword[green]{最终坍缩还是持续膨胀}
\end{itemize}

\begin{tikzpicture}[global scale = 1.3]
	
	\draw[->,black] (0, -1) -- (4, -1);
	\foreach \x in {1, 2, 3} \draw[black] (\x, -1cm - 1pt) -- (\x, -1cm + 1pt) node[anchor=north] {$\x$};
	\draw[black] (0, -1cm - 1pt) -- (0, -1cm + 1pt) node[anchor=north west] {$0$};
	\draw[black] (4.2, -1) node[anchor=west] {$\Omega_0$};
	\draw[->,black] (0, -1.5) -- (0, 4);
	\foreach \x in {-1, 0, 1, 2, 3} \draw[black] (-1pt, \x) -- (1pt, \x) node[anchor=east] {$\x$};
	\draw[black] (0, 4.2) node[anchor=east] {$\Omega_{\Lambda, 0}$};
	
	\draw[black] (0, 3) -- (3, 3);
	\foreach \x in {1, 2} \draw[black] (\x, 3cm - 1pt) -- (\x, 3cm + 1pt);
	\draw[black] (3, -1) -- (3, 3.5);
	\foreach \x in {-1, 0, 1, 2, 3} \draw[black] (-1pt + 3cm, \x) -- (1pt + 3cm, \x);

	\draw[blue] (0, 1) -- (2, -1);
	\draw[red] (0, 0) -- (3, 1.5);
	\draw[blue] (1, 0) node[anchor=north east, font=\fontsize{8}{8}] {开放};
	\draw[blue] (1, 0) node[anchor=south west, font=\fontsize{8}{8}] {封闭};
	\draw[red] (2, 1) node[anchor=north west, font=\fontsize{8}{8}] {减速};
	\draw[red] (2, 1) node[anchor=south east, font=\fontsize{8}{8}] {加速};
	
	\draw[magenta] (0, 1) .. controls (0.3, 2) and (0.5, 2.5) .. (1.3, 3);
	\draw[magenta] (0.6, 2.5) node[anchor=north west, font=\fontsize{6}{6}] {有大爆炸};
	\draw[magenta] (0.6, 2.5) node[anchor=south east, font=\fontsize{6}{6}] {无大爆炸};
	
	\draw[green] (0, 0) .. controls (1, 0.1) and (2, 0.25) .. (3, 0.45);
	\draw[green] (2.5, 0.3) node[anchor=north, font=\fontsize{6}{6}] {再坍缩};
	\draw[green] (2.5, 0.25) node[anchor=south east, font=\fontsize{6}{6}] {一直膨胀};

\end{tikzpicture}

\par 
可见$\Omega_{0}$和$\Omega_{\Lambda, 0}$是两个很关键的可观测量
\begin{cbox}
	{$\Omega_{0}$和$\Omega_{\Lambda, 0}$的观测值是多少?}
	当前关于超新星和CMB以及重子声学振荡(BAO)等多种观测已经把参数的可能取值限制在很小的范围内,$\Omega_{0} \simeq 0.3, \Omega_{\Lambda, 0} \simeq 0.7$ \\
	所以当前宇宙$k \simeq 0$,加速膨胀,存在大爆炸,持续膨胀,我们把$\Omega_{0} \simeq 0.3, \Omega_{\Lambda, 0} \simeq 0.7$叫做宇宙学标准模型,即Flat $\Lambda$CDM模型
\end{cbox}

\subsection{宇宙年龄}
\subsubsection{观测限制}
The age of the universe, 'Physics Reports', 307, (1998) 23-30, Review.
\begin{itemize}
	\item[1. ] 地质学限制:$T_{\text{地球}} \sim 50 \text{亿年}$
	\item[2. ] 核年代学研究星系盘中铀同位素相对值:大约有$T > 100 \text{亿年}$
	\item[3. ] 白矮星冷却曲线:大约有$T > 100 \text{亿年}$
	\item[4. ] 最佳方案:研究古老球状星团,得到$T \simeq 100  \textasciitilde 130 \text{亿年}$
\end{itemize}

\subsubsection{理论}
\begin{itemize}
	\item[1. ] Milne宇宙($\rho_{\text{tot}} = 0$)$a(t) = \frac{t}{t_0}, H(t) = \frac{1}{t}, v(t) = \text{const}$,宇宙从$t=0, a(0) = 0$开始匀速膨胀到$a(t_0) = 1$的状态,得到宇宙年龄$\frac{1}{H_0}$约为140亿年($h \simeq 0.7$)
	\item[2. ] 物质与辐射混合$k = \Lambda = 0$,辐射比物质衰减得快 $\Rightarrow$ 宇宙演化中大多数时间物质主导$\frac{\dot{a}}{a} = H(t) = \frac{2}{3t}$,故有宇宙年龄约为$t_0 = \frac{2}{3H_0} \simeq 93 \text{亿年}$
	\item[3. ] Flat $\Lambda$CDM模型$\Omega + \Omega_{\Lambda} = 1$ \\
	Friedmann方程$H^2 = \frac{8 \pi G}{3} \rho_{\text{tot}} = \qty(\frac{\dot{a}}{a})^2, \rho_{\text{tot}} = \rho + \rho_{\Lambda} \simeq \rho_{\text{mat}} + \rho_{\Lambda} = \rho_0 a^{-3} + \rho_{\Lambda} \Rightarrow \dot{a}^2 = \frac{8 \pi G}{3} (\rho_0 a^{-1} + \rho_{\Lambda, 0} a^{2}) = H_0^2 (\Omega_0 a^{-1} + \Omega_{\Lambda, 0} a^2)$ \\
	$\dot{a} = H_0 \sqrt{\Omega_0 a^{-1} + \Omega_{\Lambda, 0} a^2}$积分得到$\int_{0}^{T} \dd t = \frac{2}{3 H_0} \frac{\arctan \sqrt{1-\Omega_{0}}}{\sqrt{1-\Omega_{0}}} \simeq 0.964 t_H \simeq 1.3456 \times 10^{10} \text{年}$
	
\end{itemize}


\subsection{暗物质}
\par 
$\Omega_{0} = 0.3$,占宇宙30\%的非相对论物质不全是重子物质
\subsubsection{重子物质的占比}
\begin{cbox}
	{发光恒星的质量密度占$\rho_c$的多少?}
	恒星的结构理论:由恒星的温度亮度$\Rightarrow$估算出质量,若观测足够大的区域$\Rightarrow$估算区域内恒星质量密度 \\
	研究结论(Ryden, P126):$\Omega_{\text{Star}, 0} \simeq 0.003$,只占$\Omega_{0}$的1\%
\end{cbox}


\subsubsection{恒星之外的重子物质}
\par
是否所有的重子物质都以恒星形式存在?
\begin{itemize}
	\item[1. ] MACHOs(massive compact halo objects 大质量致密晕天体) \\
	它们是一些低质量的恒星(发出的光太微弱探测不到),如褐矮星($M < 0.08 M_{\text{Sun}}$)或恒星尺度的小型黑洞,这些天体一般通过引力透镜效应进行观测 \\
	但即使考虑了MACHOs,对$\Omega_{\text{Star}, 0}$的估值也低于0.005,因此可作为DM的一部分但不会是全部
	\item[2. ] 星系气体 \\
	不仅存在于各个星系内(e.g.银河系或M31星系气体的质量是恒星质量的20\%),还大量存在于星系团内(e.g.在距离银河系100Mpc处的Coma Cluster后发座星系团,其恒星总质量$M_{\text{Comma star}} \simeq 2 \times 10^{13} M_{\text{Sun}}$,但该星系团内存在大量温度极高的星际气体$T \sim 10^8 \mathrm{K}$,可用X射线望远镜看到,对这些气体的研究给出$M_{\text{Comma gas}} \simeq 2 \times 10^{14} M_{\text{Sun}}$,是恒星质量的10倍) \\
	星际气体也存在于星系、星系间之外的广阔的星际空间中,这部分气体占总重子数的85\%!但它们密度极低,温度极低,目前技术很难对它们直接测量
\end{itemize}

\par 
CMB的观测结果和原初核合成理论限制:$0.021 \leqslant \Omega_{B} h^2 \leqslant 0.025$,若取$h = 0.7$,则$\Omega_{B} \in \qty[0.043, 0.051]$,仅有4\% \textasciitilde 5\%是重子物质,剩下25\%是非重子物质

\par 
结论:
\begin{itemize}
	\item[1. ] 发光天体只占重子物质的10\%左右
	\item[2. ] 绝大多数重子物质不以恒星的形式存在,暗重子物质主要以星际气体形式存在,占总密度的4\%
	\item[3. ] 宇宙中绝大多数的物质不仅不发光,而且不是重子物质,统称为暗物质
\end{itemize}

\subsubsection{暗物质存在的其它证据}
\begin{itemize}
	\item[1. ] 星系内存在DM(首次1914年Slipher) \\
	理论上,由向心力公式和万有引力公式得到速度$v = \sqrt{\frac{G M(R)}{R}}$,当$R$较大时,$M(R) \leftarrow \text{const}$,故$v \propto R^{- \frac{1}{2}}$ \\
	则Keplerian转动,$v$将衰减,而对M31等漩涡星系的观测表明(星系旋转曲线),$R$很大时$v$趋近于常数,这表明星系中有大量暗物质,实测速度比理论速度高3倍以上$\Rightarrow M_{\text{DM}} \simeq 10 M_{\text{star}}$
	
	\item[2. ] 星系团内也存在DM(首次1933年Zwicky) \\
	仍以Coma Cluster为例,一般有3种方法可以估算出星系团的总质量:
	\begin{itemize}
		\item[1)] 质点系的位力定理可以算出$M_{\text{coma}} \simeq 2 \times 10^{15} M_{\text{Sun}}$
		\item[2)] 高温气体辐射压与星系团自引力平衡也可推算出$M_{\text{coma}} \simeq 1.3 \times 10^{15} M_{\text{Sun}}$
		\item[3)] 引力透镜,结果一致
	\end{itemize}

	\item[3. ] 结构形成:1980s人们意识到重子物质不足以提供形成当前观测结构的引力,需要引入暗物质
\end{itemize}

\par 
主流观点认为,DM是由一种新型的与重子物质相互作用极弱的非重子物质构成,通常还假设这些物质缺乏使之形成盘状结构的耗散机制$\Rightarrow$DM形成球状晕结构 \\
\resizebox{\linewidth}{!}{$
	$$
	\Omega_{\text{tot}} 
	\begin{cases}
		\Omega_{\text{mat}} \sim 0.3 \rightarrow 
		\begin{cases}
			\Omega_{\text{DM}} = 0.2 \textasciitilde 0.25 \\
			\text{核合成} \Omega_{B} = 0.04 \textasciitilde 0.05 
			\begin{cases}
				\Omega_{\text{star}} = 0.005 \textasciitilde 0.01 \\
				\Omega_{\text{non-star}} = 0.04 \textasciitilde 0.05 (\text{星际气体})
			\end{cases} 
		\end{cases} \\
		\Omega_{\text{rad}} \sim 10^{-5} \\
		\Omega_{\Lambda} \sim 0.7
	\end{cases}
	$$
$}

\subsubsection{暗物质可能是什么?}
\begin{itemize}
	\item[A] 基本粒子说:
	\begin{itemize}
		\item[1. ] 我们已知的基本粒子,但对其性质不确定,如中微子 \\
		轻中微子,作为热暗物质,但不能很好解释宇宙结构形成(只可能作为一小部分) \\
		重中微子,作为冷暗物质,大多数阶段非相对论性
		\item[2. ] 超越标准模型的基本粒子,如BSM(超对称)中最轻的超对称伙伴(LSP) \\
		LSP取决于模型,LSP是WIMPs(Weak Interacting Massive Particles)的一类 \\
		另一类WIMPs例子:额外维中的Kaluza-Klein粒子
	\end{itemize}

	\item[B] 致密天体说:
	\begin{itemize}
		\item[1. ] 原初黑洞(Primordial Black Holes),前提:形成原初黑洞原在核合成之前
		\item[2. ] MACHOs,但丰度不足以解释星系晕的形成,对其探测的主要手段是引力透镜
	\end{itemize}
\end{itemize}
\subsubsection{暗物质探测法}
\begin{itemize}
	\item[1. ] 粒子对撞机
	\item[2. ] 基于暗物质湮灭
	\item[3. ] 暗物质与固体、液体的散射:与核碰撞散射 
	$$\begin{cases}
		\text{电信号:GaAs} \\
		\text{光信号:NaI} \\
		\text{热信号:产生声子进行探测}
	\end{cases}$$
\end{itemize}