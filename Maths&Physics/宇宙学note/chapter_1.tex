\par 
宇宙学是以宇宙为研究对象的一门学科。
\par 
宇宙:“上下四方曰宇,往古来今曰宙。”——战国·尸佼
\par 
宇宙是整个时空以及它的内容,包括一切天体等物质、能量。——Wikipedia

\subsection{宇宙学研究内容}
\begin{itemize}
	\item[1] 时间与空间
		\begin{itemize}
			\item[·] 几何结构:平直or弯曲?无限or有限?
			\item[·] 宇宙静态or演化 \quad 演化过程
		\end{itemize}
	\item[2] 宇宙介质
		\begin{itemize}
			\item[·] 物质
				\begin{itemize}
					\item[·] 可见:光子、重子物质
					\item[·] 不可见:暗物质,间接方法推知存在
				\end{itemize}
			\item[·] 这些物质占整个宇宙的比例?如何影响宇宙演化?
			\item[·] 暗能量:如何发现?所占比例?如何影响宇宙演化?
			\item[·] 物质、能量占比是否随时间演化,空间中如何分布?
			\item[·] 宇宙的物质形式是否变化?恒星与星系如何形成?
		\end{itemize}
\end{itemize}

\subsection{宇宙学的特殊性}
\begin{itemize}
	\item[1] 特殊性
	\begin{itemize}
		\item[1)] 研究对象:宇宙,仅此一个,只能靠观测和计算机模拟
		\item[2)] 我们在宇宙中的位置固定
			\begin{cbox}{在哪看有什么区别?}
				光传播速度有限$\Rightarrow$视界\quad 可观测宇宙
			\end{cbox}
		\item[3)] 宇宙演化过程复杂,牵涉知识广 
		\begin{itemize}
			\item[·] 时空演化——广义相对论GR
			\item[·] 高能粒子作用(早期)——粒子与核物理、统计力学
		\end{itemize}
	\end{itemize}
	\item[2] 和天体物理、天文学关系
	\par 天文学:
	\begin{itemize}
		\item[·] 宇宙学——研究整个宇宙的行为
		\item[·] 天体物理——研究单个或少量天体、星系
	\end{itemize}
\end{itemize}

\subsection{物理宇宙学简史}
\subsubsection{阶段一(1915-1930)}
\par 
初期,健康发展,在观测方面:
\begin{itemize}
	\item 1915 Einstein GR
	\item 1917 Einstein提出第一个物理宇宙学模型(Einstein模型)
	两条假设:
	\begin{itemize}
		\item[1)] 宇宙看作充满全空间的均匀介质
		\item[2)] 宇宙整体上看来静态,引入宇宙学常数$\Lambda$
	\end{itemize}
	\item 1920初 \quad 天文学家发现宇宙膨胀,抛弃了静态模型
	\item 1923 Slipher测量十多个漩涡星云,首次发现光谱大部分有红移现象,由Doppler效应,说明星云远离\\ 同一时期,Hubble发现星云实际是河外星系,人们意识到宇宙是以星系“分子”组成的气体
	
	\item 1929 Hubble定律,星系退行速度与距离成正比
	\begin{cnote}
		Hubble定律不仅说明宇宙在膨胀而且是以保持宇宙均匀性的方式膨胀,方式唯一
	\end{cnote}

\end{itemize}
\par 
在理论方面:
\begin{itemize}
	\item 1923 Friedmann采用不含宇宙学常数的模型,发现宇宙在引力作用下可以膨胀,预言了Hubble定律
\end{itemize}

\subsubsection{阶段二(1930-1960s)}
\par
挫折期,挫折原因:
\begin{itemize}
	\item[1] Hubble定律$v = H_0 d, H_0 = 500 \mathrm{km/s/Mpc}$得$\frac{1}{H_0}\sim \text{宇宙年龄} \sim \text{20亿年}$ \\
	Friedmann模型的理论基础:均匀性宇宙和宇宙演化服从GR
	\item[2] 反推宇宙有初始时刻$\rightarrow$宇宙创生,科学向宗教靠拢
\end{itemize}
\begin{cnote}
	Hubble定律是唯一能够达到宇宙均匀性的膨胀方式
\end{cnote}
\begin{itemize}
	\item 1940s Gamov提出宇宙演化论,观点:一切物质形态都不是亘古不变的
	\item 1950s 理论预言10K左右背景光谱
	\item 1950s中后期 $H_0 \sim 50-100 \mathrm{km/s/Mpc}$,解决了宇宙年龄矛盾
	\item 1965 宇宙微波背景辐射(CMB)发现$\Rightarrow$重新重视Hubble-Friedmann-Gamov宇宙理论
\end{itemize}

\subsubsection{阶段三(1970-now)}
\par 
蓬勃发展,发现:
\begin{itemize}
	\item[1] 微波背景的各向异性
	\item[2] 宇宙加速膨胀 $\rightarrow$ 暗能量(1998年)
	\item[3] 引力波的发现
	\item[4] 暗物质的发现
	\item[5] 暴涨理论的提出
\end{itemize}