\subsection{黑夜为什么是黑的?Olbers佯谬}
\begin{itemize}
	\item 矛盾表述:
	$r$\textasciitilde$r + \dd r$球壳内恒星密度$n = \mathrm{const}$,恒星个数$N = 4 \pi r^2 n \dd r$,单颗恒星亮度$\frac{L}{4 \pi r^2}$,则球壳内总亮度$\frac{L}{4 \pi r^2} N = L n \dd r$,与距离无关 $\rightarrow$ 积分发散说明无黑夜,矛盾
	\item 结论:一个恒星密度为常数的宇宙,不可能是无穷大静态的
	\item 答案:
	\begin{itemize}
		\item[1)] 空间膨胀 $\rightarrow$ 红移 $\rightarrow$ 超出可见光谱
		\item[2)] 宇宙年龄有限 $\rightarrow$ 可观测宇宙有限
	\end{itemize}
\end{itemize}

\subsection{光(电磁波)辐射}
\par 
当前主要手段

\subsubsection{可见光}
\begin{itemize}
	\item[1] 恒星、与太阳类似的天体
	\begin{cnote}
		距离单位:1秒差距(parsec),$1 \mathrm{pc} = 3.261 \mathrm{l.y.}$
	\end{cnote}
	
	\item[2] 星系\quad 超过数十亿恒星 \\
	例如:Milky Way银河系,$3 \times 10^{11}$颗恒星,总质量$\sim 10^{12} M_{\text{Sun}}$,形态为盘状,半径12.5kpc,厚度0.3kpc,太阳距离银心8kpc,旋转周期2亿年
	
	\item[3] 星系群\quad 由星系构成,比如银河系属于本星系群(Local Galaxy Group) \\
	离MW最近的星系:大麦哲伦云,50kpc \\
	与MW大小相近的:仙女座星系,770kpc \textasciitilde 0.77Mpc
	\begin{cnote}
		星系群典型体积$\sim \mathrm{Mpc^3}$ \\
		宇宙学最常用单位,相邻星系群典型距离$1 \mathrm{Mpc} \sim 3.086\times 10^{22} \mathrm{m}$
	\end{cnote}

	\item[4] 星系团$\subset$超星系团(宇宙纤维、宇宙长城) \\
	中间会有宇宙空洞$\sim$50Mpc,当尺度大于100Mpc时,宇宙是均匀的
\end{itemize}

\subsubsection{微波}
\begin{itemize}
	\item 理论预言宇宙微波,Gamov(1950s),10K
	\item 首次发现,1965年Penzias \& Wilson,1978年获Nobel Prize
	\item 首次精确测量(1989-1993),COBE(Cosmic Background Explore)
\end{itemize}
\par 结果:
\begin{itemize}
	\item[1] 接近$2.725 \mathrm{K} \pm 0.001 \mathrm{K}$黑体谱
	\item[2] 近似各向同性,各方向温度几乎相同
	\item[3] CMB有一定的各向异性$10^{-5} \mathrm{K}$,意义:可推知宇宙早期情况,与宇宙起源密切相关
\end{itemize}

\subsubsection{射电波}
\par 
氢原子的超精细21cm谱线 $\rightarrow$ 宇宙中存在大量中性气体

\subsubsection{X射线}
\par 
高温气体($10^7 \mathrm{K}$)放出,观测星系团,90\%的星系团高温气体不在星系中

\subsubsection{红外线与$\gamma$射线}
\par 
对宇宙学用处不大,多用于天体物理

\subsection{其它辐射}
\subsubsection{中微子}
\par
特点:轻,与其它粒子相互作用弱(探测也就较难)
\par 
但高能中微子(如超新星爆发产生的)可以探测到
\par 
大爆炸产生的中微子(难测)

\subsubsection{引力波}
\par 
由双致密星(BH或者中子星)产生GW
\par
大爆炸产生原初引力波,通过其对微波背景极化的影响而间接探测

\subsubsection{宇宙射线}
\par 
可以在它与地球大气层或探测器碰撞时被探测到。但由于他们并非来自很远处,因此对宇宙学用处不大。

\subsubsection{未知的其它高能物质(例如:暗物质)}
\par 
用各种暗物质实验仪器来探测

\begin{cbox}[red]{宇宙学原理}
	在大尺度上($\geqslant 100 \mathrm{Mpc}$)看,宇宙是均匀且各向同性的。
	\begin{itemize}
		\item 大尺度观测 $\rightarrow$ 均匀性
		\item CMB $\rightarrow$ 早期宇宙($t \simeq 37 \text{万年}$)几乎均匀且各向同性
	\end{itemize}
\end{cbox}