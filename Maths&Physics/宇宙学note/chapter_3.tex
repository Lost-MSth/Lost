\subsection{再论宇宙学原理(Cosmology Principle) \& Hubble定律}
\begin{cnote}
	宇宙学原理:在大尺度上($\geqslant 100 \mathrm{Mpc}$)看,宇宙是均匀且各向同性的。
\end{cnote}
\par
起初是纯假设,现在有部分观测支持:
\begin{itemize}
	\item[1)] 星系分布的大尺度结构,均匀性
	\item[2)] CMB 早期宇宙($t \simeq 37 \text{万年}$)几乎均匀且各向同性 \\
	$10^{-5}$各向异性 $\rightarrow$ 提供关键信息,与理论模型对比
\end{itemize}

\begin{cbox}
	{早期宇宙满足CP,晚期是否满足?变化宇宙怎么时刻保持其均匀且各项同性?}
	要保持宇宙学原理,则宇宙中任何位置相邻组元(如星系)之间的间距必须按相同比例变化,变化速率可随时间变化,但必须对任意空间点相同
\end{cbox}

\begin{tikzpicture}[global scale = 0.7]
	\draw (-3, 1.5) circle(0) node[anchor=north] {$t_1$:};
	
	\draw (-2, 1.5) circle(0.1) node[anchor=north] {A};
	\draw (-1, 1.5) circle(0.1) node[anchor=north] {B};
	\draw (0, 1.5) circle(0.1) ;
	\draw (1, 1.5) circle(0.1) node[anchor=north] {C};
	\draw (2, 1.5) circle(0.1) node[anchor=north] {D};
	
	\draw[->,blue] (-2.5, 1) -- (3.5, 1);
	\foreach \x in {-2, -1, 0, 1, 2, 3} \draw[blue] (\x, 1cm + 1pt) -- (\x, 1cm - 1pt) node[anchor=north] {$\x$};
	\draw[blue] (3.5, 1) node[anchor=west] {$r$};
	
	\draw[->,red] (-2.5, 0) -- (3.5, 0);
	\foreach \x in {-2, -1, 0, 1, 2, 3} \draw[red] (\x, 1pt) -- (\x, -1pt) node[anchor=north] {$\x$};
	\draw[red] (3.5, 0) node[anchor=west] {$x$};
	
	
	
	\draw (-5, -1.5) circle(0) node[anchor=north] {$t_2$:};
	
	\draw (-4, -1.5) circle(0.1) node[anchor=north] {A};
	\draw (-2, -1.5) circle(0.1) node[anchor=north] {B};
	\draw (0, -1.5) circle(0.1) ;
	\draw (2, -1.5) circle(0.1) node[anchor=north] {C};
	\draw (4, -1.5) circle(0.1) node[anchor=north] {D};
	
	\draw[->,blue] (-4.5, -2) -- (6.5, -2);
	\foreach \x in {-4, -3, -2, -1, 0, 1, 2, 3, 4, 5, 6} \draw[blue] (\x, -2cm + 1pt) -- (\x, -2cm - 1pt) node[anchor=north] {$\x$};
	\draw[blue] (6.5, -2) node[anchor=west] {$r$};
	
	\draw[->,red] (-4.5, -3) -- (6.5, -3);
	\foreach \x in {-2, -1, 0, 1, 2, 3} \draw[red] (\x * 2, -3cm + 1pt) -- (\x * 2, -3cm - 1pt) node[anchor=north] {$\x$};
	\draw[red] (6.5, -3) node[anchor=west] {$x$};
	
	\draw[->] (-3.4, 1.1) .. controls (-4, 1) and (-4.5, 0.3) .. (-4.8, -1.3);
\end{tikzpicture}

\begin{itemize}
	\item 可建立一个相对于宇宙静止的坐标系$r$
	\item 除物理坐标系还可建立随宇宙不断膨胀的坐标系$x$(共动坐标系Comoving Coordinate)
\end{itemize}
\par 
可以看出物理距离等于共动距离乘以一个只与时间有关的因子$a(t)$,称为尺度因子
\begin{equation}
	r(t) = a(t) x
	\label{math:1}
\end{equation}
其中,$a(t)$描述宇宙在$t$时刻膨胀的速率,宇宙无中心,点点相同
\par 
结论:若CP始终成立,尺度因子只能是时间的函数,与空间无关;式\eqref{math:1}。
\begin{cnote}
	真实宇宙并非精确满足宇宙学原理,对均匀背景的偏离研究是宇宙学的重要研究方向
\end{cnote}

\subsection{Hubble定律及红移}
\subsubsection{Hubble定律}
\par 
由物理坐标系表达式\eqref{math:1}可计算出物理速度:
\begin{equation}
	v(t) = \frac{\dd a}{\dd t} x = \frac{\dot{a}}{a} a x = \frac{\dot{a}}{a} r \equiv H(t) r
\end{equation}
其中$H(t) = \frac{\dot{a}}{a}$为哈勃参数,当$t = t_0$,即当前时刻时,$H(t_0) = H_0$称为哈勃常数,是重要的待测参数。

\subsubsection{红移与尺度因子的关系}
\par 
哈勃定律中物理速度是通过红移测量计算的,当相对速度不大($v \ll c$),有$Z = \frac{v}{c}$
\par 
红移定义:$Z = \frac{\lambda_r - \lambda_e}{\lambda_e}$ $\Rightarrow$ $t_r$与$t_e$时刻光波长之比$\frac{\lambda_r}{\lambda_e} = 1 + Z$
\par 
考虑相距$\dd r$,相对速度$\dd v$的两个星系,$\dd v = \frac{\dot{a}}{a} \dd r$,红移为$\frac{\dd \lambda}{\lambda} = Z = \frac{\dd v}{c} = \frac{\dot{a}}{a} \frac{\dd r}{c} = \frac{\dot{a}}{a} \dd t = \frac{\dd a}{a}$,故$\lambda \propto a$,则$t_r$与$t_e$时刻的波长比:
\begin{equation}
	\frac{\lambda_r}{\lambda_e} = \frac{a(t_r)}{a(t_e)} = 1 + Z
	\label{math:2}
\end{equation}
若$t_r = t_0, a(t_0) \equiv 1$,则当前红移值$Z = \frac{1}{a(t_e)} - 1$,对于大爆炸($a = 0$),红移$Z \rightarrow \infty$;对于当前光,$Z = 0$

\begin{cnote}
	推导红移公式\eqref{math:2}时有假设两星系间距小,但实际上,用GR可严格导出对任意距离、任意宇宙几何的情况,波长(红移)与尺度因子都满足此公式
\end{cnote}

\begin{cbox}
	{宇宙膨胀的含义?}
	人不膨胀,因为化学键占主导;星系内,星体间引力作用占主导……实际上宇宙膨胀力密度很小
\end{cbox}
\begin{cbox}
	{退行速度超光速?}
	没有光到达,没有因果关系,不破坏因果律
\end{cbox}

\subsubsection{红移畸变效应}
\par 
星系的红移除了包括宇宙学红移,即星系发出的光子波长随背景宇宙的膨胀而拉长导致的红移,还包括星系本动速度造成的多普勒红移。假定我们拥有一个准确的宇宙学模型可以将星系红移正确地转化为星系与我们的距离,那么星系本动速度造成的多普勒红移会在星系的真实距离上叠加一个不可忽略的系统误差。导致观测到的星系三维位置其中一个维度 (视线方向上) 产生了“畸变”,导致我们观测到的星系空间分布呈现了各向异性的特征,即我们所说的“红移畸变效应”。
\par 
研究意义:星系本动速度大小反映了宇宙结构增长的快慢,所以通过测量星系红移巡天中的红移畸变效应可以帮助我们了解宇宙结构增长的历史,从而更好地甄别宇宙学模型和准确地限制模型参数。



\subsection{Friedmann方程}
\par 
引力势能:$V = - \frac{G M m}{r}$
\par 
模型:宇宙中均匀且各向同性地分布着质量密度为$\rho$的介质,由宇宙学原理知$\rho(t) = \sum_i \rho_i(t)$,任取一点作为原点建立物理坐标系,考虑处于$\vb*{r}$处质量为$m$的粒子,在此坐标系下受力为半径$r$的球给予的引力$M = \frac{4}{3} \pi r^3 \rho \Rightarrow V(r) = - \frac{4}{3} \pi G r^2 \rho m$ 
\par 
质点动能$T = \frac{1}{2} m \dot{r}^2$,故总能量$E = T + V = \frac{1}{2} m \dot{r}^2 - \frac{4}{3} \pi G r^2 \rho m = \text{const}$
\par 
$r(t) = a(t) x$是质点与观者之间的物理距离,代入得$E = \frac{1}{2} m \dot{a}^2 x^2 - \frac{4}{3} \pi G \rho m a^2 x^2$,化简得$\frac{2E}{m a^2 x^2} = (\frac{\dot{a}}{a})^2 - \frac{8}{3} \pi G \rho$
\par 
定义$\frac{2 E}{m x^2} =- k c^2$,其中$c$为光速,则代入得到\cword{Friedmann方程}:
\begin{equation}
	(\frac{\dot{a}}{a})^2 = \frac{8}{3} \pi G \rho - \frac{k c^2}{a^2} = (H(t))^2
	\label{math:3}
\end{equation}
其中$H(t)$为哈勃参数。

\par 
决定$a(t)$演化的有:宇宙介质的总密度$\rho$和常数$k$(其实就是曲率)
\par
若要满足宇宙学原理,空间曲率只有三种可能,如表\ref{table:1}所示。
\begin{table}[!h]
	\tiny
	\centering
	\begin{tabular}{|c|c|c|c|c|}
		\hline
		曲率$k$ & 几何             & 三角形内角和 & 圆周长 & 宇宙类型  \\ \hline
		$> 0$   & Spherical球空间   & $> \pi$      & $> 2 \pi r$  & Close \\ \hline
		$= 0$   & Flat平空间        & $= \pi$      & $= 2 \pi r$  & Flat  \\ \hline
		$< 0$   & Hyperbolic双曲空间 & $< \pi$      & $< 2 \pi r$  & Open  \\ \hline
	\end{tabular}
	\caption{三种空间性质对比}
	\label{table:1}
\end{table}

\par 
对Friedmann方程的讨论:
\begin{itemize}
	\item[1. ] $\rho(t)$是质量密度,其它地方可能会说是能量密度,差个$c^2$,没问题
	\item[2. ] $t = t_0$时,尺度因子$a(t_0) = 1$,$H_0^2 = \frac{8}{3} \pi G \rho_0 - k \quad(c = 1)$
	\item[3. ] 当$k = 0$,$\rho(t) = \frac{3 H_0^2}{8 \pi G} \equiv \rho_c(t)$称为临界密度 \\
	$\rho_c(t)$的物理意义:临界密度是使得宇宙刚好是平直几何的密度值 \\
	当前时刻$\rho_c(t_0) = \frac{3 H_0^2}{8 \pi G} = 1.88 h^2 \times 10^{-26} \mathrm{kg/m^3} = 2.78 h^2 \times 10^{11} M_{\text{Sun}} \mathrm{/Mpc^3}$,其中$H_0 = 100 h \quad \mathrm{km \cdot s^{-1} \cdot Mpc^{-1}}$,$h$是由实验待测的参数 \\
	$\rho_c(t)$还相当于200L的体积里有一个质子,而200L液态水里有$\sim 10^{29}$个质子 \\
	\begin{table}[!h]
		\tiny
		\centering
		\resizebox{\linewidth}{!}{
		\begin{tabular}{|c|c|c|c|c|}
			\hline
			时间   & 1990s                              & 现在                                                & 2010             & 2019                                                                \\ \hline
			项目 & \begin{tabular}[c]{@{}c@{}}Hubble space\\ Telescope key project\end{tabular}  & SHOES & Planck Satellite & \begin{tabular}[c]{@{}c@{}}HOLiCOW\\ arXiv: 1907.04869\end{tabular} \\ \hline
			$h$    & $0.72 \pm 0.08$      & $0.75 \pm 0.03$             & $0.673 \pm 0.012$   & $0.733 \pm 0.017$        \\ \hline
		\end{tabular}}
		\caption{$h$的测量}
		\label{table:2}
	\end{table}
	$h \sim 0.7$相当于100Mpc外的星体退行速度$\sim$ 7000km/s
	
	\item[4. ] 利用$\rho_c(t)$可定义一个无量纲密度常数$\Omega(t) = \frac{\rho(t)}{\rho_c(t)} \Rightarrow \rho = \Omega \rho_c = \Omega \frac{3 H^2}{8 \pi G}$ \\
	可以将Friedmann方程改写为$H^2 = H^2 \Omega - \frac{k}{a^2}$,即$\Omega = 1 + \frac{k}{H^2 a^2}$,$\Omega = 1$的宇宙称为临界密度宇宙 \\
	一般宇宙介质含多种类型,可定义相应的密度参数$\Omega = \Omega_{\text{mat}} + \Omega_{\text{rad}} + \cdots$ \\
	有时候也定义曲率的“密度参数”为$\Omega_k = - \frac{k}{a^2 H^2}$,从而Friedmann方程改写为$\Omega + \Omega_k = 1$
\end{itemize}


\subsection{热力学与流体方程}
\par 
单纯由Friedmann方程还不够解出$a(t)$,若将宇宙演化看作孤立系统,就可用热力学第一定律导出$\rho(t)$的另一个方程
\par 
考虑一个单位半径的共动球体,物理体积为$V = \frac{4}{3} \pi (ax)^3 = \frac{4}{3} \pi r^3$,总质量$m = \rho V$,总能量$E = m c^2 = \frac{4}{3} \pi a^3 \rho c^2$
\par 
该体积内宇宙介质在膨胀过程中满足热一:
\begin{equation}
	\dd E = T \dd S - p \dd V \overset{绝热}{=} - p \dd V 
\end{equation} 
而且我们有:
\begin{equation}
	\dv{E}{t} = \frac{4 \pi}{3} \qty(3 a^2 \dv{a}{t} \rho c^2 + a^3 \dv{\rho}{t} c^2 )
\end{equation}
以及$\dv{V}{t} = 4 \pi a^2 \dv{a}{t}$,综合三式可得:
\begin{equation}
	\dot{\rho} + 3 \frac{\dot{a}}{a} \qty(\rho + \frac{p}{c^2}) = 0
	\label{math:4}
\end{equation}
此方程即\cword{流体方程},这里可以发现我们又引入了新的未知量:压强$p(t)$
\par 
讨论:
\begin{itemize}
	\item[1. ] 导致宇宙介质能量变化的原因有两个:宇宙膨胀导致密度降低(括号里第一项)和体积膨胀介质通过压强对外做功(括号里第二项)
	\item[2. ] 由于宇宙学原理,宇宙各处压强均等,则$p(t)$不是$\vb*{x}$的函数且不存在压力$\grad{p} = 0$ \\
	做功损失的能量 $\Rightarrow$ 引力势能 $\Rightarrow$ 能量守恒
	\item[3. ] 目前有两个方程:Friedmann方程\eqref{math:3}和流体方程\eqref{math:4},联立可得\cword{加速度方程}
	\begin{equation}
		\frac{\ddot{a}}{a} = - \frac{4}{3} \pi G \qty(\rho + \frac{3p}{c^2})
		\label{math:6}
	\end{equation}
	注意该方程不独立,不含曲率$k$ \\
	$p>0, \rho >0 \Rightarrow \ddot{a} < 0$,故宇宙减速膨胀,要加速膨胀要求$p<0$(引入暗能量的动机)
\end{itemize}

\subsection{物态方程及宇宙介质的分类}
\par 
宇宙学中大多时候考虑最简单、最重要的\cword{物态方程}:
\begin{equation}
	p = \omega \rho c^2
	\label{math:5}
\end{equation}
其中,$\omega$是无量纲常量,不同介质$\omega$不一样:
$$
\omega = 
\begin{cases}
	0, \qq{非相对论性物质、无压物质(尘埃)} \\
	\frac{1}{3}, \qq{辐射、相对论性物质} \\ 
	-1, \qq{真空能或宇宙学常数(暗能量)}
\end{cases}
$$
\par 
将物态方程\eqref{math:5}代入流体方程\eqref{math:4},得到$\frac{\dot{\rho}}{\rho} = - 3 (1 + \omega) \frac{\dot{a}}{a}$,则有$\rho(a) \propto a^{-3 (1 + \omega)}$,对应上述三种情况我们有:
$$
\left\{
\begin{aligned}
	\rho_{\text{mat}} &\propto a^{-3} \qq{尘埃} \\
	\rho_{\text{rad}} &\propto a^{-4} \qq{辐射} \\
	\rho_{\Lambda} &= \text{const} \qq{宇宙真空能} \\
\end{aligned}
\right.
$$
\par 
下面给出一些物质粒子的分类:
\begin{table}[!h]
	\centering
	\scriptsize
	\begin{tabular}{|c|c|c|c|c|}
		\hline
		物质粒子    & 静质量/MeV         & 电荷 & 按速度大小分类                                                                                             & 宇宙介质类型                \\ \hline
		质子p     & 938.27          & 1  & \multirow{3}{*}{\begin{tabular}[c]{@{}c@{}}1. 最重要的重\\ 子物质\\ 2. 当前宇宙中\\ 最重要的非相\\ 对论性物质\end{tabular}} & \multirow{3}{*}{无压物质} \\[9pt] \cline{1-3}
		中子n     & 939.57          & 0  &                                                                                                     &                       \\[9pt] \cline{1-3}
		电子e     & 0.5110          & -1 &                                                                                                     &                       \\[9pt] \hline
		中微子$\mathrm{\nu}$   & $< 3\times 10^{-7}$ & 0  & 相对论性                                                                                                & 辐射                    \\ \hline
		光子$\mathrm{\gamma}$ & 0               & 0  & 相对论性                                                                                                & 辐射                    \\ \hline
		暗物质?    & ?               & 0  & \begin{tabular}[c]{@{}c@{}}1. 非量子类\\ 2. 相对论或\\ 非相对论性\\ 依赖于模型\end{tabular}                           & 依赖于模型                 \\ \hline
		暗能量     & -               & -  & -                                                                                                   & 能量                    \\ \hline
	\end{tabular}
	\caption{物质粒子分类表}
\end{table}

\subsubsection{宇宙介质的分类}
\par 
在宇宙学中,不同的宇宙介质是按照组成它们的基本粒子的性质分类的,可以根据粒子运动快慢分为:
\begin{itemize}
	\item[1. ] 非相对论性:$v \ll c$,此时$E = m_0 c^2 + \frac{p^2}{2 m_0}$ \\
		对于多粒子体系,可以根据$m_0 c^2$和$k_B T$之间的关系来区分,非相对论性有$m_0 c^2 \gg k_B T$
	\item[2. ] 相对论性:$m \simeq 0, v \simeq c, E \simeq pc$,例如光子
\end{itemize}

\begin{itemize}
	\item[1. ] 非相对论物质:特点:相互作用不频繁,可认为$p=0$无压 \\
	代表:重子物质 \\
	宇宙 $\leftarrow$ 原子 $\leftarrow$ 核子 $\leftarrow$ 夸克(p=uud, n=udd) \\
	稳定的重子只有质子和中子,故它们最重要 
	\begin{cnote}
		单个中子不稳定,$\tau \sim 15 \mathrm{min}$,与质子组成核的中子是稳定的
	\end{cnote}
	在宇宙学中,有时也把电子视为重子物质 \\
	宇宙电中性 $\Rightarrow$	电子数 = 质子数 \\
	在当前宇宙中,重子物质均为非相对论性的,非相对论性条件:对电子$T \ll 6 \times 10^9 \mathrm{K}$;对质子$T \ll 1 \times 10^{13} \mathrm{K}$ \\
	宇宙中重子物质原子大约有$\frac{3}{4} \mathrm{H}$、$\frac{1}{4} \mathrm{He}$和其它
	
	\item[2. ] 辐射物质:广义地说辐射是指所有相对论性的物质(例:重子物质若运动较快,如宇宙早期,也视作辐射)\\ 
	当前宇宙辐射的代表:光子和中微子
\end{itemize}

\subsubsection{光子}
\par 
$m = 0$且$E = \hbar \omega$,可以与其它粒子互相作用。若一盒光子气体到达平衡温度$T$,那么有如下结论:
\begin{itemize}
	\item[1. ] 平衡时频率介于$f$\textasciitilde$f$+$\dd f$,能量密度$\varepsilon(f) \dd f =\frac{8 \pi h}{c^3} \frac{f^3 \dd f}{\ee^{h f/k_B T } - 1}$,其中玻尔兹曼常数$k_B = 8.619 \times 10^{-5} \mathrm{eV/K} = 1.381 \times 10^{-23} \mathrm{J/K}$ \\
	$x \equiv \frac{h f}{k_B T}$,则$\varepsilon(f) = \frac{8 \pi (k_B T)^3}{c^3 h^2} \frac{x^3}{\ee^x - 1} \equiv \varepsilon(x)$,极值点$x_{\text{max}} \simeq 2.81$ \\
	$f_{\text{$\varepsilon$-peak}} \simeq 2.81 \frac{k_B T}{h}$,相应的光子能量$h f \simeq 2.81 k_B T$,则总能量主要是由$3 k_B T$左右的光子贡献
	
	\item[2. ] 光子气体的总能量密度$\varepsilon_{\gamma} = \int_{0}^{\infty} \varepsilon(f) \dd f = \frac{8 \pi (k_B T)^4}{c^3 h^3} \int_{0}^{\infty} \frac{x^3 \dd x}{\ee^x - 1} = \frac{8 \pi^5 (k_B T)^4 }{15 c^3 h^3}$ \\
	$\Rightarrow$ 温度为$T$时的光子气体总能量密度$\varepsilon_{\gamma} = \alpha T^4$,其中$\alpha \simeq 7.565 \times 10^{-16} \mathrm{J \cdot m^{-3} \cdot K^{-4}}$ \\
	又由$\rho_{\text{rad}} \propto a^{-4}$,以及$\varepsilon_{\text{rad}} = \rho_{\text{rad}} c^2 \propto a^{-4}$和$\varepsilon_{\text{rad}} \propto T^4$得到$T \propto a^{-1}$
	
	\item[3. ] 光子数密度按频率分布,频率$f$\textasciitilde$f$+$\dd f$,$n(f) = \frac{\varepsilon(f)}{h f} = \frac{8 \pi}{c^3} \frac{f^2}{\ee^x - 1} =  \frac{8 \pi (k_B T)^2}{c^3 h^2} \frac{x}{\ee^x - 1} $ \\
	极大值处$x \simeq 1.59$,$f_{\text{n-peak}} \simeq 1.59 \frac{k_B T}{h}$
	
	\item[4. ] 光子总粒子数密度$n_{\gamma} = \int_0^{\infty} n(f) \dd f = \frac{8 \pi (k_B T)^3}{c^3 h^3} \int_{0}^{\infty} \frac{x^2 \dd x}{\ee^x - 1} = \beta T^3$,其中$\beta \simeq 2.4041 \qty(\frac{k_B}{\hbar c})^3 \frac{1}{\pi^2} \simeq 2.029 \times 10^{7} \mathrm{m^{-3} \cdot K^{-3}}$,所以$n_{\gamma} \propto a^{-3}$
	
	\item[5. ] 光子平均能量$E_{\text{mean}} = \frac{E_{\text{tot}}}{N} = \frac{\varepsilon_{\gamma}}{n_{\gamma}} = \frac{\alpha}{\beta} T \simeq T \cdot 3.745 \times 10^{-23} \mathrm{J/K} \simeq 2.7 k_B T$
	
\end{itemize}

\par 
\textbf{小结:}
\begin{itemize}
	\item[1)] 平衡时仍有各频率的光子
	\item[2)] 不同频率光子的能量密度分布由黑体函数决定
	\item[3)] $\varepsilon(f)$在$f_{\text{$\varepsilon$-peak}} \simeq 2.81 \frac{k_B T}{h}$处取极大值 $\Rightarrow$ 对光子气体能量贡献最多的是$E = 2.81 k_B T$附近的光子,平均能量约为$2.7 k_B T$ 
\end{itemize}
\par 
\textbf{核心:}
\begin{itemize}
	\item[1)] 光子总能量密度$\varepsilon_{\gamma} = \alpha T^4$,总光子数密度$n_{\gamma} = \beta T^3$,单个光子平均能量$E_{\text{mean}} = \frac{\varepsilon_{\gamma}}{n_{\gamma}} \simeq 2.7 k_B T$
	\item[2)] $ \left.
	\begin{aligned}
		\varepsilon_{\text{CMB}} &\propto T^4 \\
		\rho_{\text{CMB}} &\propto a^{-4}
	\end{aligned}
	\right\}$
	$\Rightarrow$ $T_{\text{CMB}} \propto a^{-1}$
	
	\item[3)] $n_{\gamma} \propto a^{-3}$,对于非相对论物质,可证明也有数密度$n_{\text{matter}} \propto a^{-3}$ \\
	proof: $\rho_{\text{mat}} \propto \varepsilon_{\text{mat}} = n_{\text{mat}} E_{\text{sm}}$,其中$E_{\text{sm}}$为单个粒子的平均能量,故$\rho \propto a^{-3} \Rightarrow n \propto a^{-3}$
\end{itemize}

\begin{cbox}[red]
	{总粒子数密度是非常重要的,因为大多数时候粒子数守恒} 
	粒子间的相互作用可以导致粒子间的相互转化,因此当相互作用不频繁时$N$守恒;另一方面,相互作用很频繁时,$N$也应当守恒;只有当平衡破坏,有相互作用但不平衡时,$N$才不守恒。故大多数时候$N$守恒。 \\
	此时唯一导致粒子数密度变化的因素就是物理体积的变化$n = \frac{N}{V} \propto \frac{N}{a^3 x^3} \propto \frac{1}{a^3}$ \\
	\textbf{辐射和尘埃存在很多不同,但它们的数密度随$a$的演化行为相同,$n \propto a^{-3}$}
\end{cbox}

\begin{cnote}
	重要常数:常用单位$1 \mathrm{eV} = 1.602 \times 10^{-19} \mathrm{J}$ \\
	$k_B = 1.381 \times 10^{-23} \mathrm{J/K} = 8.619 \times 10^{-5} \mathrm{eV/K}$,$h \simeq 4.136 \times 10^{-15} \mathrm{eV \cdot s}$ \\
	$\alpha = \frac{\pi^2 k_B^4}{15 \hbar^3 c^3} = 7.565 \times 10^{-16} \mathrm{J \cdot m^{-3} \cdot K^{-4}}$(光子气体的总能量密度$\varepsilon_{\gamma} = \alpha T^4$) \\
	$\beta = \frac{2.4041}{\pi^2} \qty(\frac{k_B}{c \hbar})^3 = 2.029 \times 10^7 \mathrm{m^{-3} \cdot K^{-3}}$(光子气体的总光子数密度$n_{\gamma} = \beta T^3$)
\end{cnote}

\begin{itemize}
	\item[A] 应用:CMB \\
	观测表明,当前的CMB与温度为$T_0 = 2.7655 \pm 0.0006 K$的黑体辐射谱符合较好
	\begin{itemize}
		\item[1)] $\varepsilon_{\text{CMB}} = \alpha T_0^4 = 4.175 \times 10^{-14} \mathrm{J/m^3} = 0.2606 \mathrm{MeV/m^3}$ \\
		$\rho_{\text{CMB}} = \frac{\varepsilon_{\text{CMB}}}{c^2} = 4.639 \times 10^{-31} \mathrm{kg/m^3}$,进一步算出CMB光子的密度参数 \\
		$\Omega_{\text{CMB}} = \frac{\rho_{\text{CMB}}}{\rho_{\text{critical}}} = \frac{4.639 \times 10^{-31}}{1.88 h^2 \times 10^{-26}} \simeq 2.47\times 10^{-5} h^{-2} \quad (h \simeq 0.7)$ \\
		可见光子对临界密度的贡献很小
		\begin{cbox}
			{CMB光子和宇宙中恒星发出的光子哪个多?}
			当前星系亮度密度$\Psi \simeq 1.7 \times 10^{8} L_{\text{Sun}}/\mathrm{Mpc^{-3}} \simeq 2.2 \times 10^{-33} \mathrm{W/m^3}$,$\varepsilon_{\text{starlight}} \sim 0.006 \mathrm{MeV/m^3} \sim 2.3\% \varepsilon_{\text{CMB}}$(粗略估计) \\
			更为细致的研究表明:$\frac{\varepsilon_{\text{starlight}}}{\varepsilon_{\text{CMB}}} \simeq 0.1$,早期宇宙该值更小,故在计算宇宙光子能量密度时可近似忽略非CMB光子
		\end{cbox}
	
		\item[2)] 光子数密度为$n_{\text{CMB}} \simeq 4.107 \times 10^8 \mathrm{/m^3}$,相当于411个$\mathrm{/cm^3}$ \\
		$\Rightarrow$ CMB光的平均能量$E_{\text{mean}} \simeq 2.7 k_B T \simeq 6.344 \times 10^{-4} \mathrm{eV}$,平均频率$f_{\text{mean}} = \frac{E_{\text{mean}}}{h} \simeq 1.534 \times 10^{11} \mathrm{Hz}$,波长$\lambda = \frac{c}{f} \simeq 2 \mathrm{mm}$(微波波段)
	\end{itemize}

	\item[B] 应用:中微子(共三种$\nu_{e}$、$\nu_{\mu}$、$\nu_{\tau}$),它们的质量都很小$m < 3 \times 10^{-7} \mathrm{MeV/c^2}$ \\
	相互作用弱,很难探测,假设$m=0$ $\stackrel{\text{理论}}{\Longrightarrow}$ $\Omega_{\nu} \simeq 0.681 \Omega_{\text{CMB}}$ \\
	因此当前宇宙中辐射的密度参数$\Omega_{\text{rad}}{}_0 = \Omega_{\nu}{}_0 + \Omega_{\text{CMB}}{}_{0} \simeq 4.15 \times 10^{-5} h^{-2}$
\end{itemize}

\subsubsection{宇宙学常数$\Lambda$(最简单的暗能量模型)}
\begin{cbox}
	{何故引入$\Lambda$?}
	在宇宙学中的一个重要可观测常数:减速参数$q_0 \equiv - \frac{\ddot{a}(t_0)}{a(t_0)} \frac{1}{H_0} = - \frac{a(t_0) \ddot{a}(t_0)}{\qty(a(t_0))^2}$ \\
	于1990年代对\Rmnum{1}${}_a$型超新星研究发现$q_0 < 0, \ddot{a} > 0$,说明宇宙加速膨胀 \\
	然而,由加速度方程\eqref{math:6}可知无论是matter($p=0$)还是radiation($p = \frac{1}{3} \rho c$),必有$\ddot{a} < 0$,矛盾
\end{cbox}

\par 
为了解释宇宙加速膨胀,最简单的方式:假设宇宙不仅由尘埃和辐射组成,还有一种(等效)物质密度为常数的分量$\rho_{\text{tot}} = \rho + \rho_{\Lambda} = \rho_{\text{mat}} + \rho_{\text{rad}} + \rho_{\Lambda}$,其中$\rho_{\Lambda} \equiv \frac{\Lambda}{8 \pi G}$,相应的密度参数$\Omega_{\Lambda} = \frac{\Lambda}{8 \pi G} \frac{8 \pi G}{3 H^2} = \frac{\Lambda}{3 H^2}$是含时的
\begin{cnote}
	我们总是用$\rho = \rho_{\text{mat}} + \rho_{\text{rad}} $表示广义物质,其中$\rho_{\text{mat}}$代表非相对论物质,$\rho_{\text{tot}} = \rho + \rho_{\Lambda}$为宇宙介质总质量密度
\end{cnote}
宇宙密度参数$\Omega_{\text{tot}} \equiv \frac{\rho_{\text{tot}}}{\rho} = \Omega_{\text{mat}} + \Omega_{\text{rad}} + \Omega_{\Lambda}$,由于$\rho_{\Lambda}$为常数,由$\dot{\rho_{\Lambda}} + 3 \frac{\dot{a}}{a} \qty(\rho_{\Lambda} + \frac{p_{\Lambda}}{c^2}) = 0$ $\Rightarrow$ $p_{\Lambda} = - \rho_{\Lambda} c^2$,有负压,可以解释加速膨胀

\par 
含$\Lambda$的Friedmann方程$H^2 = \frac{8 \pi G}{3} \rho_{\text{tot}} - \frac{k c^2}{a^2}$或者写成$1 = \frac{8 \pi G}{3 H^2} \rho_{\text{tot}} - \frac{k c^2}{a^2 H^2} = \Omega_{\text{tot}} - \frac{k c^2}{a^2 H^2} = \Omega_{\text{tot}} + \Omega_k$

\par 
密度参数与宇宙几何的关系
$$
\left\{
\begin{aligned}
	0 < \Omega_{\text{tot}} < 1 \Rightarrow \Omega_k > 0 , k < 0 \qq{开放宇宙} \\
	\Omega_{\text{tot}} = 1 \Rightarrow \Omega_k = 0 , k = 0 \qq{平坦宇宙} \\
	\Omega_{\text{tot}} > 1 \Rightarrow \Omega_k < 0 , k > 0 \qq{封闭宇宙} 
\end{aligned}
\right.
$$
\
\begin{cbox}
	{$\Lambda$项可解释加速膨胀,物理上对应什么?}
	一种可能的解释是$\Lambda$是真空能,但是由粒子物理所预言的真空能远远大于$\Lambda$的观测值,该矛盾叫宇宙学常数问题 \\
	其它解释:精质(quintessence)模型,$\Lambda(t)$随时间缓变,物态方程一般只需要$\omega< - \frac{1}{3}$就可以解释加速膨胀
\end{cbox}